$K = \R, \C$.
\\ geg. lineare Rekursion mit konstanten Koeffizienten der Ordnung $k$:
\[x_n = c_1x_{n-1} + \dots + c_kx_{n-k} + g(n)\] für alle $n \geq k +1$

Zu gegebenen Anfangswerten $(a_1, \dots, a_k) \in K^k$  existiert genau eine Lösung $(x_1, x_2, \dots) \in k^{\N}$ (= Menge aller unendlichen Folgen mit Einträgen aus $K$) mit $x_i = a_i$ für $1 \leq i \leq k$

$l : \begin{cases}
(a_1, \dots, a_k) \mapsto l(a) = \text{ Lösung mit Anfangswert } a_1, \dots, a_k \\ K1 k \to K^{\N}
\end{cases}$

\subsubsection*{Beispiel}

$x_n = 2x_{n-1} + 3x_{n-3}$, Ordnung 3, $(c_1 = 2, c_2 = 0, c_3 = 3)$

z.B. Anfangswerte $a=(0, 1, 2)$.
\\ $l(a) = (0, 1, 2, 4, 11, 28, 68, \dots)$

\subsection{Satz} %2.3
\begin{enumerate}
	\item Geg. homogene lin. Rek mit konst.Koeff., Ordnung k
	
	($R_k) x_n=c_1x_{n-1}+...+c_kx_{n-k}$
	
	Die Lösungen von ($R_k$) bilden einen k-dim. Unterraum L von $K^{\N}, l:K^k \rightarrow L \subseteq K^{\N} $ Vektorraum-isom.(d.h. bijektive lin. Abb.)
	
	\item Inhom. lin. Rekursion
	
	$(R)x_n=c_1x_{n-1}+...+c_kx_{n-k}+g(n),~ n\geq k+1$
	
	Zugehörige hom. Rekursion sei: ($R_k)-g(n)=0$ setzen.
	
	Ist $b=(b_1,b_2,...)$ eine spezielle Lösung von (R), so ist die Menge aller Lösungen von (R) gegeben durch:
	
	$(b+L)=\{(b_1+y_1,b_2+y_2,...):(y_1,y_2) \in L \}$ \qquad L Lösungsraum der $(R_k)$
	
	[(R) in "unendlicher" Matrixform:
	
	$\begin{pmatrix}
	-c_k & -c_{k-1} & ... & -c_1 & 1 & 0 &... & & \\
	0&-c_k & -c_{k-1} & ... & -c_1 & 1 & 0 &... &\\
	0&0&-c_k & -c_{k-1} & ... & -c_1 & 1 & 0 &... \\
	&.&&.&&.&&.&\\
	&.&&.&&.&&.&\\
	&.&&.&&.&&.&\\
	\end{pmatrix}
	\begin{pmatrix}
	x_1\\
	x_2\\
	x_3\\
	.\\
	.\\
	.\\
	\end{pmatrix} = 
	\begin{pmatrix}
	g(k+1)\\
	g(k+2)\\
	g(k+3)\\
	.\\
	.\\
	.\\
	\end{pmatrix}
	\begin{matrix}
	-c_kx_1-c_{k-1x_2}-...-c_1x_k+x_{k+1}= g(k+1)\\
	.\\
	.\\
	.
	\end{matrix}$
	]
\end{enumerate}
\subsubsection*{Beweis}
Wie für LGS ind er lin. Algebra.

\begin{itemize}
	\item[1.Ziel:] Finde geschlossene Form für die Lösungen von ($R_k$), d.h. $x_n=f(n)$
	\item[2.Ziel:] Finde geschlossene Form für die Lösungen von (R) 
\end{itemize}

Wie kommt man zur geschlossenen Beschreibung der Lösungen in Lösungsraum L von ($R_n$)?

Basis von L bestimmen. l bildet Basis von $k^l$ auf Basis von L ab.

z.B. $l(e_1),...,l(e_k),$ wobei $e_1, ..., e_k$ kan. Basis von $K^k$

$l(e_1)= (1*0*...*\underbrace{0}_k, \underbrace{c_k}_{k+1},\underbrace{c_1c_k}_{k+2}, \underbrace{c_1^2c_k+c_2c_k}_{k+3},...)$ sieht nicht schön aus!

Wir suchen eine schönere Bassis von L!

Beschreibung hom. lin. Rek. durch lin. Schieberegister:

%TODO ADD REGISER GRAPHIC
\fbox{$x_{n-1}$} $\rightarrow$ \fbox{$x_{n-2}$} $\rightarrow$ ... $\rightarrow$ \fbox{$x_{n-k}$} $\rightarrow$ \qquad \qquad Zeitpunkt $n-k$ (Zum Zeitpunkt 1: Anfangswerte in Regisern)

Beschreibung des Überangs der Registerbelegung zum Zeitpunkt $n-k$ zum nächsten Zeitpunkt $n-k+1$ wird beschrieben durch llin. Abb.:

$\alpha \begin{pmatrix}
x_{n-k}\\
.\\
.\\
.\\
x_{n-1}
\end{pmatrix}=
\begin{pmatrix}
x_{n-k+1}\\
.\\
.\\
.\\
x_{n-1}\\
c_1x_{n-1}+...+c_kx_{n-k}
\end{pmatrix}=
~,~
\alpha \underbrace{\begin{pmatrix}
	x_{1}\\
	.\\
	.\\
	.\\
	x_{k}
	\end{pmatrix}}_{Anfangsw}=~
\begin{pmatrix}
x_{2}\\
.\\
.\\
.\\
x_{k+1}
\end{pmatrix}, \quad
\alpha^2 \begin{pmatrix}
x_{1}\\
.\\
.\\
.\\
x_{k}
\end{pmatrix}
\begin{pmatrix}
x_{3}\\
.\\
.\\
.\\
x_{k+2}
\end{pmatrix}, \quad
\alpha^m \begin{pmatrix}
x_{1}\\
.\\
.\\
.\\
x_{k}
\end{pmatrix}
\begin{pmatrix}
x_{m+}\\
.\\
.\\
.\\
x_{m+k}
\end{pmatrix}
$

$\mathcal{P}$ kan. Basis von $K^k$ \qquad Darstellungsmatrix $A_{\alpha}^{\mathcal{P}}$:

$\begin{pmatrix}
0\\.\\.\\0\\c_k
\end{pmatrix}=
\alpha\begin{pmatrix}
1\\0\\.\\0
\end{pmatrix},\alpha\begin{pmatrix}
0\\1\\.\\0
\end{pmatrix},...,
\alpha\begin{pmatrix}
0\\0\\.\\1
\end{pmatrix}$

$A_i=A_{\alpha}^\mathcal{B} = \begin{pmatrix}
0&1&0&.&.&.&0\\
0&0&1&.&.&.&0\\
0&0&0&.&.&.&0\\
.&.&.&.&.&.&.\\
0&0&0&.&.&.&1\\
c_k&c_{k-1}&c_{k-2}&.&.&.&c_1
\end{pmatrix}\qquad \qquad 
A\begin{pmatrix}
x_1\\.\\.\\.\\x_k
\end{pmatrix}=\begin{pmatrix}
x_2\\x_3\\.\\.\\.\\x_k\\x_{k+1}
\end{pmatrix} $ \qquad
\fbox{$x_{k+1} = c_kx_1+...+c_1x_k$}

Wunsch : Bringe $A$ auf Diagonalgestalt (d.h. finde Basis von $K^k$ aus Eigenvektoren zu $\alpha$!)

Bestimme Eigenwerte von $A$ (d.h. von $\alpha$). Charakteristisches Polynom $\det(t\cdot E_k - A)$ berechnen.

$\det(t E_k - A) = \det \begin{pmatrix}
t & -1 & & & &\\ 
&  t &-1  & & &\\
& & &\ddots  & & \\
& & & & t & -1 & \\
-c_k & -c_{k-1} & \cdot  && -c_2 & t -c_1 
\end{pmatrix}
= \footnote{Entwicklung nach der 1. Spalte}
\dots 
= t^k - c_1t^{k-1} - \dots - c_{k-1}t -c_k$

Eigenwerte von $A$ = Lösungen von
\[ t^k - c_1t^{k-1} - \cdot - c_{k-1}t - c_k = 0 \]

Sei $d$ eine Nullstelle des charakteristischen Polynoms, also Eigenwert von $A$.

Sei $\sigma \neq v = \begin{pmatrix}
v_1 \\ \vdots \\ v_k
\end{pmatrix}$ ein zugehöriger Eigenvektor. 

$\begin{pmatrix}
v_2 \\ \vdots \\ c_1v_k + \dots + c_kv_1
\end{pmatrix}
= Av
= dv
= \begin{pmatrix}
d \cdot v_1 \\ d \cdot v_2 \\ \vdots \\ d \cdot v_k
\end{pmatrix}
$\qquad
$\begin{array}{l}
v_2 = d \cdot v_1 \\
v_3 = d^2 \cdot v_1 \\
\vdots \\
v_k = d^{k-1}v_1
\end{array}$

Durch $v_1$ ist $v$ eindeutig bestimmt. Eigenraum von $\alpha$ zu Eigenwert $d$ ist 1-dimensional. Setze $v_1 = 1$

Eigenvektor zu $d$ : $v = \begin{pmatrix}
1 \\ d \\ d^2 \\ \vdots \\ d^{k-1}
\end{pmatrix}$

$v^t = (1, d, d^2, \dots, d^{k-1})$
\\$l(v^t) = (1, d, d^2, \dots, d^{k-1}, d^k, d^{k+1}, \dots) \in L$

$\begin{pmatrix}
v_{m-1} \\ \vdots \\ v_{m+k}
\end{pmatrix}
= A^m \begin{pmatrix}
v_1 \\\vdots\\v_k
\end{pmatrix}
= A^m \begin{pmatrix}
1 \\ d\\ \vdots \\d^{k-1}
\end{pmatrix}
= d^m \begin{pmatrix}
1 \\ d \\\vdots \\d^{k-1}
\end{pmatrix}
= \begin{pmatrix}
d^m \\ d^{m+1} \\\vdots \\d^{m+k-1}
\end{pmatrix}
$


















