$K = \R, \C$.
\\ geg. lineare Rekursion mit konstanten Koeffizienten der Ordnung $k$:
\[x_n = c_1x_{n-1} + \dots + c_kx_{n-k} + g(n)\] für alle $n \geq k +1$

Zu gegebenen Anfangswerten $(a_1, \dots, a_k) \in K^k$  existiert genau eine Lösung $(x_1, x_2, \dots) \in k^{\N}$ (= Menge aller unendlichen Folgen mit Einträgen aus $K$) mit $x_i = a_i$ für $1 \leq i \leq k$

$l : \begin{cases}
(a_1, \dots, a_k) \mapsto l(a) = \text{ Lösung mit Anfangswert } a_1, \dots, a_k \\ K1 k \to K^{\N}
\end{cases}$

\subsubsection*{Beispiel}

$x_n = 2x_{n-1} + 3x_{n-3}$, Ordnung 3, $(c_1 = 2, c_2 = 0, c_3 = 3)$

z.B. Anfangswerte $a=(0, 1, 2)$.
\\ $l(a) = (0, 1, 2, 4, 11, 28, 68, \dots)$

\subsection{Satz} %2.3
%TODO Thomas


Wunsch : Bringe $A$ auf Diagonalgestalt (d.h. finde Basis von $K^k$ aus Eigenvektoren zu $\alpha$!)

Bestimme Eigenwerte von $A$ (d.h. von $\alpha$). Charakteristisches Polynom $\det(t\cdot E_k - A)$ berechnen.

$\det(t E_k - A) = \det \begin{pmatrix}
t & -1 & & & &\\ 
&  t &-1  & & &\\
& & &\ddots  & & \\
& & & & t & -1 & \\
-c_k & -c_{k-1} & \cdot  && -c_2 & t -c_1 
\end{pmatrix}
= \footnote{Entwicklung nach der 1. Spalte}
\dots 
= t^k - c_1t^{k-1} - \dots - c_{k-1}t -c_k$

Eigenwerte von $A$ = Lösungen von
\[ t^k - c_1t^{k-1} - \cdot - c_{k-1}t - c_k = 0 \]

Sei $d$ eine Nullstelle des charakteristischen Polynoms, also Eigenwert von $A$.

Sei $\sigma \neq v = \begin{pmatrix}
v_1 \\ \vdots \\ v_k
\end{pmatrix}$ ein zugehöriger Eigenvektor. 

$\begin{pmatrix}
v_2 \\ \vdots \\ c_1v_k + \dots + c_kv_1
\end{pmatrix}
= Av
= dv
= \begin{pmatrix}
d \cdot v_1 \\ d \cdot v_2 \\ \vdots \\ d \cdot v_k
\end{pmatrix}
$\qquad
$\begin{array}{l}
v_2 = d \cdot v_1 \\
v_3 = d^2 \cdot v_1 \\
\vdots \\
v_k = d^{k-1}v_1
\end{array}$

Durch $v_1$ ist $v$ eindeutig bestimmt. Eigenraum von $\alpha$ zu Eigenwert $d$ ist 1-dimensional. Setze $v_1 = 1$

Eigenvektor zu $d$ : $v = \begin{pmatrix}
1 \\ d \\ d^2 \\ \vdots \\ d^{k-1}
\end{pmatrix}$

$v^t = (1, d, d^2, \dots, d^{k-1})$
\\$l(v^t) = (1, d, d^2, \dots, d^{k-1}, d^k, d^{k+1}, \dots) \in L$

$\begin{pmatrix}
v_{m-1} \\ \vdots \\ v_{m+k}
\end{pmatrix}
= A^m \begin{pmatrix}
v_1 \\\vdots\\v_k
\end{pmatrix}
= A^m \begin{pmatrix}
1 \\ d\\ \vdots \\d^{k-1}
\end{pmatrix}
= d^m \begin{pmatrix}
1 \\ d \\\vdots \\d^{k-1}
\end{pmatrix}
= \begin{pmatrix}
d^m \\ d^{m+1} \\\vdots \\d^{m+k-1}
\end{pmatrix}
$


















