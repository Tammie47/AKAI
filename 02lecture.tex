\subsection{Prinzip des doppelten Abzählens}
\label{subsec:doppAbzaehlPrinzip}

$A, B$ endliche Mengen.
$S \subseteq A \times B$

Für jedes $a \in A$ sei $n_a = \abs{\aset{b \in B : (a, b) \in S}}$
\\Für jedes $b \in B$ sei $m_b = \abs{\aset{a \in A : (a, b) \in S}}$

Dann \[ \sum_{a \in A}n_a = \sum_{b \in B} m_b = \abs{S} \]


\subsubsection*{Speziell}

Angenommen $n_a = n$ für alle $a \in A$ und $m_b = m$ für alle $b \in B$.
%
\[\abs{A}\cdot n = \abs{B} \cdot m \]

\subsection[Beispiel : Handshaking-Lemma]{Beispiel}
Gegeben endlicher Graph. 

Graph $G$: Menge von Knoten (Ecken, \emph{vertex}), Menge von Kanten (\emph{edge}), jede Kante hat zwei Endknoten.
%
$G$ sei ohne Schleifen.

Dann: Anzahl der Knoten, von denen eine ungerade Anzahl von Kanten ausgeht, ist gerade. (Handshaking - Lemma)

$V=$ Menge der Knoten
\\$E=$ Menge der Kanten
\\$S \subseteq V \times E = \aset{(v, e), v \in V, e \in E, v \text{ ist Endknoten von } e}$
\\$v \in V : n_v =$ Anzahl der Kanten, die von $v$ ausgehen ($n_v =$ Grad von $v$)

\ref{subsec:doppAbzaehlPrinzip} : 
$\abs{S} = \sum_{v\in V} n_v = \sum_{e \in E} 2 = 2 \abs{E}$

$V = V_g \;\dot{\cup}\; V_u$,
\\$v \in V_g \gdw n_v$ gerade,
$v \in V_u \gdw n_v$ ungerade

$\sum_{v\in V_u}n_v + \sum_{v \in V_g}n_v = 2 \cdot \abs{E} \Rightarrow \sum_{v \in V_u}n_v$ gerade Zahl
$\Rightarrow \abs{V_u}$ ist gerade.

\subsection{Schubfachprinzip}
\label{subsec:schubfachprinzip}

Werden $m$ Objekte auf $n$ ''Schubfächer'' verteilt, so enthält (mind.) ein Schubfach mindestens $\left\lceil\frac{m}{n}\right\rceil$ Objekte. ($\lceil x \rceil = $ kleinste ganze Zahl $\geq x$)

\subsubsection*{Beweis}

$\left\lceil \frac{m}{n} \right\rceil - 1 < \frac{m}{n}$.
Angenommen Aussage ist falsch. D.h.. alle Schubfächer enthalten höchstens $\left\lceil \frac{m}{n} \right\rceil -1$ Objekte.

$\Rightarrow m \leq n \cdot \left(\left\lceil \frac{m}{n} \right\rceil -1\right) < n \cdot \frac{m}{n} = m $ %TODO lightning


\subsection[Beispiel: Anwendung Schubfachprinzip]{Beispiele} 

\begin{enumerate}
	\item Gegeben sein ein Quadrat der Seitenlänge 70cm. Kann man 50 Punkte so im Quadrat verteilen, dass je zwei Punkte mindestens Abstand von 15 cm haben?
	
	Zerlege das Quadrat in 49 Teilquadrate der Seitenlänge 10cm. Wenn man 50 Punkte im Quadrat verteilt, so enthält man nach \ref{subsec:schubfachprinzip} mindestens eines der kleinen Quadrate 2 Punkte.
	Maximalabstand in kleinem Quadrat ist durch die Diagonale gegeben, beträgt $\approx 14.14$ cm $< 15$ cm.
	
	\item
	Gegeben 6 Punkte in der Ebene, keine 3 auf einer Geraden. Je 2 Punkte durch Kante verbinden (vollst. Graph)
	
	Färbe Kanten in jeweils einer von 2 Farben (rot, grün).
	Dann gibt es ein einfarbiges Dreieck. 
	
	Gegeben sei irgendeine Färbung. Von den von einem Knoten $P$ ausgehenden 5 Kanten haben mindestens 3 die gleiche Farbe (\ref{subsec:schubfachprinzip}).
	
	Angenommen rote Kanten : $PQ, PR, PS$
	
	Falls $QR$ rot, so $PQR$ rotes Dreieck. O.B.d.A $QR$ grün.
	
	Analog: $QS$ und $RS$ o.B.d.A grün. Dann $QRS$ grün.
	
	\textbf{Spezialfall des Satzes von Ramsey (1930)}.
	
	Gegeben vollständiger Graph $G$ (alle Knoten sind durch Kanten verbunden).
	Färbe Knoten mit $f$ Farben. Wähle $r \in \N$
	
	Ist die Anzahl der Knoten von $G$ genügend groß, so existieren $r$ Knoten in $G$, so dass alle Kanten zwischen diesen $r$ Knoten die gleiche Farbei haben.
	(oben: $f=2, r=3$).
	
	(Beweis: Diekert, Kufsteiner, Rosenberger, Bd.1, Satz 6.29)
	
	
\end{enumerate}


\subsection*{Elementare Abzählprobleme (Mathe I)}

$n, k \in \N_0$

$[n]_0 := 1; k \geq 1: [n]_k := n \cdot (n-1) \cdot \dots \cdot (n-k + 1)$.
Für $k > n$ ist $[n]_k=0$

$[n]_n = n \cdot (n-1) \cdot 2 \cdot 1 = n!$ ($n$ Fakultät)

$\binom{n}{k} = \begin{cases}
	0 & k > n \\
	\frac{n!}{k!(n-k)!} & k \leq n
\end{cases} \;\;= \frac{[n]_k}{k!}$ Binomialkoeffizient.

$\binom{n}{k} = \binom{n}{n-k}$, $n, k \geq 1 : \binom{n}{k} = \binom{n-1}{k} + \binom{n-1}{k-1}$

$(a+b)^n = \sum_{k=0}^{n}\binom{n}{k}a^kb^{n-k} $ Binomialsatz.
\; Für $a=b=1$ : $ 2^n = \sum_{k=0}^{n}\binom{n}{k}  $ 































