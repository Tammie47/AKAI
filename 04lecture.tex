\subsection{Satz} %1.13

$A, B$ Mengen, $\abs{A}=k, \abs{B} = n$.

\begin{enumerate}
	\item
	Anzahl aller Abbildungen $A \to B$ : $n^k$
	
	\item
	Anzahl aller injektiven Abbildungen $A \to B$: $[n]_k$
	
	\item
	Anzahl aller bijektiven Abbildungen $A \to B$: 
	$\begin{cases}
	0 & n \neq k \\
	n! & n = k
	\end{cases}$
	
	Insbesondere: Anzahl der Permutationen auf $\aset{1, \dots, n} = n!$
	
	\item
	Anzahl aller surjektiven Abbildungen $A \to B$ : $S(k,n)\cdot n!$
	
\end{enumerate}

\subsubsection*{Beweis}

\begin{enumerate}
	\item $\underbrace{\underset{\underset{\in B}{\downarrow}}{a_1}, \dots, \underset{\underset{\in B}{\downarrow}}{a_k}}_{\text{jew. } n \text{ Mögl.}}$ : Insgesamt $n^k$
	
	\item
	Wie in a), aber ein schon gewähltes Bild darf nicht wieder gewählt werden:
	
	$n \cdot (n-1) \cdot \dots \cdot (n-k+1) = [n]_k$
	
	\item
	Bijektive Abbildung $A \to B$ existiert genau dann, wenn $\abs{A} = \abs{B}$
	
	$n \neq k$: 0 bijektive Abbildungen
	\\$n = k$: Anzahl bijektiver Abbildungen $\stackrel{b)}{=} [n]_k = n!$
	
	\item
	
	$A_1 \underset{\underset{b_1}{\downarrow\downarrow\downarrow}}{\aset{\dots}} \;\;\;
	A_2 \underset{\underset{b_2}{\downarrow\downarrow\downarrow}}{\aset{\dots}} \;\;\;
	\qquad
	A_n \underset{\underset{b_2}{\downarrow\downarrow\downarrow}}{\aset{\dots}} \;\;\;
	 \underset{\underset{B}{}}{A} \;\;\;
	$
	
	$A_i = \emptyset, A_i \cap A_j = \emptyset$ für $i \neq j$, 
	$A_1 \cup \dots \cup A_n = A$.
	
	Zu jeder solchen Partition von $A$ gehören $n!$ surjektive Abbildungen $A \to B$, nämlich zu jeder Anordnung der $b_1, \dots, b_n$
	
\end{enumerate}

\subsection{Satz}
$\abs{A}=k, \abs{B}=n$

Anzahl aller surjektiven Abb. $A \rightarrow B$ ist $\sum_{j=0}^{n-1}(-1)^j \binom{n}{j}(n-j^k)$

\subsubsection*{Beweis:}
Sei $B = \{b_1,...,b_n  \}$

\textit{S} = Menge aller surj. Abb. $A \rightarrow B$,

\textit{T} = Menge aller Abb.  $A \rightarrow B$,

$T_i = \{\alpha: A \rightarrow B: b_i \notin \alpha (A)  \}, i=1,...n$

Dann $S=T\backslash(T_1\cup ... \cup T_n)$

$\abs{T} = n^k$. Zu berechnen: $\abs{T_1 \cup ... \cup T_n}$ Mit 1.3 %TODO ref

$T_i$ besteht aus allen Abb. $A\rightarrow B\backslash\{b_i\}$

$\abs{T_i} = \underbrace{=}_{a)} (n-1)$ mit $n= \binom{n}{1}$  Mögl. für i

i<j: $\abs{T_i \cap T_j} = (n-2)^k$ mit $\binom{n}{2}$ Mögl. für i,j

i<j<l: $\abs{T_i \cap T_j \cap T_kl} = (n-3)^k$ mit $\binom{n}{3}$ Mögl. für i,j,l

$\abs{T_1 \cup ...\cup T_n} = \sum_{j=1}^{n}(-1)^{j+1} \sum_{1\leq i_1<...<i_j\leq n}\abs{T_{i_1}\cap ... \cap T_{i_j}} = \sum_{j=1}^{n}(-1)^{j+1} \binom{n}{j}(n-j)^k$

$\abs{S}= \abs{T}-\abs{T_1 \cup ... \cup T_n}$

$= n^k-\sum_{j=1}^{n}(-1)^{j+1}\binom{n}{j}(n-j)^{k}$

$= n^k+\sum_{j=1}^{n}(-1)^{j}\binom{n}{j}(n-j)^{k}$

$= \sum_{j=0}^{n-1}(-1)^{j}\binom{n}{j}(n-j)^{k}$

\subsection{Korollar} %1.15

\[ S(k,n) = \frac{1}{n!}\cdot \sum_{j=0}^{n-1}(-1)^j\binom{n}{j}(n-j)^k \]

\subsubsection*{Beweis}

Folgt aus 1.13.d) + 1.14. %TODO ref


\subsection{Korollar}
\begin{enumerate}
	\item $\sum_{j=0}^{n-1}(-1)^j\binom{n}{j}(n-j)^n = n!$
	
	\item Ist $n>k$, so ist $\sum_{j=0}^{n-1}(-1)^j\binom{n}{j}(n-j)^k=0$
\end{enumerate}


\subsection{Satz} %1.17
$l, m, n \in \N_0$:

\begin{enumerate}
	\item (Obere Summation)
	
	$\sum_{k=0}^{n}\binom{k}{n} = \binom{n+1}{m+1} $
	
	\item (Parallele Summation)
	
	$\sum_{k=0}^{n}\binom{m+k}{k} = \binom{m+n+1}{n} $
	
	\item (Vandermonde-Identität)
	
	$ \sum_{k=0}^{l}\binom{m}{k}\cdot\binom{n}{l-k} = \binom{m+n}{l} $   
	
\end{enumerate}

\subsubsection*{Beweis}

Kombinatorisch.

\begin{enumerate}
	
	\item 
	$\abs{A}= n+1, A = \aset{1, \dots, n+1}$.
	
	Zähle die $(m+1)$-elementigen Teilmengen von $A$. 
	1.9 %TODO ref
	: $\binom{n+1}{m+1}$
	
	$\mathcal{P}_{m+1}(A) = $ Menge aller $(m+1)$-elementigen Teilmengen von $A$
	
	Für $m \leq k \leq n$ : 
	$\mathcal{B}_k = \aset{X \in \mathcal{P}_{m+1}(A) : \text{ größtes Element ist } k+1  }  $
	
	
	$\mathcal{P}_{m+1}(A) = \;\dot{\cup}_{k=m}^n\mathcal{B}_k\;$ %TODO schön
	
	
	$\abs{\mathcal{B}_k} \stackrel{1.9}{=} %TODO ref
	 \binom{k}{m} $

	$\binom{n+1}{m+1} = \abs{\mathcal{P}_{m+1}(A)} 
	                  = \sum_{k=m}^{n}\abs{\mathcal{B}_k}
	                  = \sum_{k=m}^{n}\binom{k}{m} $
	                  
	 \item
	 $A = \aset{1, \dots, m+ n + 1}$
	 
	 $\abs{\mathcal{P}_n(A)} \stackrel{1.9}{=} %TODO ref
	 \binom{m + n + 1}{n}
	 $                 
	 
	 Zu jeder $n$-elementigen Teilmenge $B$ von $A$ gibt es ein größtes Element $k$, das \textbf{nicht} in $B$ liegt. 
	 
	 Klar: $k \geq m+1$
	 
	 $m + 1 \leq k \leq m + n + 1$:
	 $\mathcal{C}_k = \aset{X \in \mathcal{P}_n(A) : k \text{ ist das größte Element von } A \text{, das nicht in } X \text{ liegt}}$
	 
	 $\mathcal{P}_n(A) = \;\dot{\cup}_{k=m+1}^{m+n+1}\mathcal{C}_k\;$ %TODO schön
	 
	 $\abs{\mathcal{C}_k} = \binom{k-1}{k-m-1} $
	 
	 $\binom{m+n+1}{n} = \abs{\mathcal{P}_n (A)}
	                   = \sum_{k=m+1}^{m+n+1}\abs{\mathcal{C}_k}
	                   = \sum_{k=m+1}^{m+n+1}\binom{k-1}{k-m-1}
	                   = \sum_{k=0}^{n}\binom{m+k}{k}
	                    $
	 
	                  	
	\item
	
	$A = \aset{1, \dots, m+n}$, $\mathcal{P}_l(A)$
	
	$A = \underbrace{\aset{1, \dots, m}}_{A_1} \;\dot{\cup}\; \underbrace{\aset{m+1, \dots, n}}_{A_2} $
	
	 $l$-elementige Teilmenge von $A$ = Vereinigung von $k$-elementiger Teilmenge von $A_1$ und $l-k$-elementigen Teilmenge von $A_2$ ($k\leq l$)
	
	$\binom{m+n}{l} = \abs{\mathcal{P}(A)} = \sum_{k = 0}^{l} \binom{m}{k} \cdot \binom{n}{l-k}$
	
\end{enumerate}


