\section{Geordnete Mengen} % 3

\subsection{Definition}

\begin{enumerate}
	\item $M$ Menge, $\preceq$ Relation auf $M$.
	
	$\preceq$ heißt \emph{Ordnungsrelation} (oder partielle Ordnung) $\Leftrightarrow$ 
	\begin{itemize}
		\item $a \preceq a$ für alle $a\in M$ (Reflexivität)
		\item $a \preceq b$ und $b \preceq a \Rightarrow a=b$ (Antisymmetrie)
		\item $a \preceq b$ und $b \preceq c$ $\Rightarrow a \preceq c$ (Transitivität)
	\end{itemize}
	
	Schreibweise $a \prec b$, falls $a \preceq b $ und $a \neq b$
	
	\item
	$a, b \in M$. 
	
	$[a, b] = \aset{c \in M : a \preceq c \text{ und } c \preceq b}$ \emph{Intervall}
	\\ (Falls $a \npreceq b$, so ist $[a, b] = \emptyset$ wegen der Transitivität von $\preceq$)

	\item
	$a, b \in M$. 
	
	$a$ heißt \emph{Vorgänger} von $b$, $a\lessdot b$, falls $a \prec b$ und $[a, b] = \aset{a, b}$
	
	\item
	 $(M, \preceq)$ heißt \emph{total} oder \emph{vollständig geordnet} (falls $M$ endlich: \emph{Kette}), falls für alle $a, b \in M$ gilt: $a \preceq b$ oder $b \preceq a$ 
		
\end{enumerate}


\subsection{Beispiel}
 
 \begin{enumerate}
	 
	 \item $(\R, \leq)$, $\leq$ normale Kleiner-Gleich-Relation, total geordnet
	 \\ $(\Z, \leq)$ ebenso. 
	 
	 In $(\R, \leq)$ hat kein Element einen Vorgänger! 
	 \\In $(\Z, \leq)$ hat jedes Element $z$ einen Vorgänger, nämlich $z-1$
	 
	 \item $(\N, |)$, $|$ Teilerrelation, partielle Ordnung (nicht vollständig)
	 
	 \item $A$ Menge, $\mathcal{P}(A) = \aset{X : X \subseteq A}$, Teilmengenrelation $\subseteq$ ist partielle Ordnung auf $\mathcal{P}(A)$ (nicht total, falls $\abs{A} \geq 2$)

	 \item $M = \aset{0, 1}^n = \underset{\leftarrow n \rightarrow}{\aset{0,1} \times \dots \times \aset{0,1}}$ alle 0,1 - Folgen der Länge $n$.
	 
	 \emph{Lexikographische Ordnung} auf $M$: 
	 \\ $(a_1, \dots, a_n) \leq (b_1, \dots, b_n) \Leftrightarrow (a_1, \dots, a_n) = (b_1, \dots b_n)$
	 oder es existiert $1 \leq i \leq n$ mit $a_j = b_j$ für alle $j < i$ und $a_i = 0$ und $b_i = 1$ 
	 
 \end{enumerate}

Veranschaulichung endlicher partiell geordneter Mengen durch \emph{Hasse-Diagramme}.
Gerichteter Graph, Knoten = Elemente von $M$, gerichtete Kante zwischen $a$ und $b$ $\Leftrightarrow$ $a\lessdot b$

Lasse Richtungspfeile weg, Richtung ist immer von unten nach oben. 

\subsection{Beispiel} %3.3
	%TODO Zeichnungen ?
\begin{enumerate}
	\item $M = \aset{a, b, c, d, e, f}$
	
	$a \lessdot c, d \lessdot f, d \lessdot e$,
	$a \leq e, a \nleq d$
	\\ $[a, f] = \emptyset, [b, f] = \aset{b, d, f}$
	\\ Vorgänger von $e$: $c,d$
	
	\item $M = \aset{1, 2, \dots, 6}$, normale Kleiner-Gleich-Relation.
	
	\item $M = \aset{1, 2, 3, 4, 6, 12} = \mathbb{T}_{12}$ = Menge aller Teiler von $12$, Teilerrelation
	
	\item $A = \aset{1, 2, 3}, M = \mathcal{P}(A), \subseteq$, $\abs{M} = 2^3 = 8$
	
\end{enumerate} 

	\subsection{Definition} %3.4
	
	$(M, \preceq)$ geordnete Menge.
	
	$A(M, \preceq) = \aset{f : M \times N \to \C : f(a, b) = 0 \text{, falls } a \npreceq b}$ 
	\emph{Inzidenzalgebra} zu $(M, \preceq)$
	
	Ist $M$ endlich, so kann man jedes $f \in A(M, \preceq)$ als Matrix schreiben.
	
	$M = \aset{a_1, \dots, a_n}$
	
	$\begin{array}{c|ccccc}
			& a_1	& \dots & a_j 	& \dots & a_n \\\hline
	a_1  	&		&		&\vdots	\\
	\vdots  &		&		&\vdots \\
	a_i		&\dots 	& \dots & f(a_i, a_j) \\
	\vdots  & \\
	a_n		& \\
	\end{array}$
	
\subsection{Satz} %3.5
%TODO Thomas part

\subsection{Definition} %3.6

Zeta-Funktion $\zeta \in A(M, \preceq)$  ist definiert durch

$\zeta(a, b) = \begin{cases}
1 & \text{ falls } a \preceq b \\
0 & \text{ falls } a \npreceq b
\end{cases}$
\\ (beschreibt $\preceq$)

\subsubsection*{Beispiel}

$M = \mathbb{T}_{12}$, $|$

$\zeta \leftrightarrow
%
\begin{array}{c|cccccc|}
	& 1 & 2 & 3 & 4 & 6 & 12 \\\hline
1	& 1 & 1 & 1 & 1 & 1 & 1 \\
2 	& 0 & 1 & 0 & 1 & 1 & 1 \\
3 	& 0 & 0 & 1 & 0 & 1 & 1 \\
4	& 0 & 0 & 0 & 1 & 0 & 1 \\
6	& 0 & 0 & 0 & 0 & 1 & 1 \\
12 	& 0 & 0 & 0 & 0 & 0 & 1
\end{array}
%
$

\subsection{Satz und Definition} % 3.7
%TODO Thomas










