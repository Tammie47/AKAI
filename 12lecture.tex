\subsection{Beispiele} %3.8

\begin{enumerate}
	\item 
	$M$ total geordnet. $a_1 < a_2 < \dots < a_n$
	
	$\zeta \leftrightarrow
	\begin{array}{c|cccc|}
	a_1 	& 1 & 1 & \dots & 1 \\
	\vdots  &	& 1 & \dots & 1 \\
	\vdots  & 	& 0 & \ddots & \\
	a_n		&   &	&		& 1 \\ %TODO hier | unterbrechen
			& a_1 & \dots &\dots & a_n
	\end{array}$
	
	$\mu(a, b) = \begin{cases}
	1 & \text{falls } a = b  \\
	-1 & \text{falls } a \lessdot b \\
	0 & \text{sonst}
	\end{cases}$
	
	$\mu \leftrightarrow
	\begin{array}{c|cccc|}
		a_1 	& 1 & -1 &  & 0 \\
		\vdots  &	& \ddots & \ddots &  \\
		\vdots  & 	& 0 & \ddots & -1\\
		a_n		&   &	&		& 1 \\ %TODO hier | unterbrechen
				& a_1 & \dots &\dots & a_n
		\end{array}$
		
	Nach 3.7 %TODO ref
	ist zu zeigen: $\mu(a, b) = 0$, falls $a \neq b$, $a$ kein Vorgänger von $b$.
	
	$a, b, a \neq b$. $M$ total geordnet: $a < b$ oder $b < a$.
	
	Falls $b < a$, so ist $ a \nleq b$, also $\mu(a, b) = 0$.
	
	Sei $a < b$. $a = b_0 \lessdot b_1 \lessdot \dots \lessdot b_{m-1} < b_m = b, m \geq 2$
	
	$\mu(a, b) \stackrel{3.7}{=} %TODO ref
	- \sum_{i=0}^{m-1}\mu(a, b_i) 
	= - \mu(a, b_{m-1}) . \sum_{i=0}^{m-2}\mu(a, b_i)
	\stackrel{3.7}{=} %TODO ref
	- \mu(a, b_{m-1}) + \mu(a, b_{m-1})	= 0$
	
	
	\item
	
	$M = (\mathbb{T}_{20}, |)$,
	$\mathbb{T}_{20} = \aset{n \in \N : n | 20} = \aset{1, 2, 4, 5, 10, 20}$
	
	$\zeta \leftrightarrow
	\begin{pmatrix}
	1 & 1 & 1 & 1 & 1 & 1 \\
	0 & 1 & 1 & 0 & 1 & 1 \\
	0 & 0 & 1 & 0 & 0 & 1 \\
	0 & 0 & 0 & 1 & 1 & 1 \\
	0 & 0 & 0 & 0 & 1 & 1 \\
	0 & 0 & 0 & 0 & 0 & 1
	\end{pmatrix}$
	
	Berechnung von $\mu$:
	
	$\mu(a, b) = 0$, falls $a\nmid b$ 
	\\$\mu(a, a) = 1$ für alle $a \in \mathbb{T}_{20}$
	\\$\mu(1, 2) = \mu(2, 10) = \mu(5, 10) = \mu(2, 4) = \mu(10, 20) = \mu(4, 20) = -1$
	\\$\mu(1, 4) = \mu(5, 20) = 0$, da $[1, 4], [5, 20]$ Ketten, dann a).
	\\ $\mu(1, 10) \stackrel{3.7}{=} %TODO ref
	-\mu(1, 2) - \mu(1, 2) - \mu(1, 5) = - 1 + 1 + 1 = 1$
	\\ Analog $\mu(2, 20) = 1$
	\\ $\mu(1, 20) \stackrel{3.7}{=} %TODO ref
	-\mu(1, 1) - \mu(1, 2) - \mu(1, 4) - \mu(1,5) - \mu(1, 10) = -1 + 1 -0 +1 -1 = 0$
	
	
	$\mu \leftrightarrow
		\begin{pmatrix}
		1 & -1 & 0 & -1 & 1 & 0 \\
		0 & 1 & -1 & 0 & -1 & 1 \\
		0 & 0 & 1 & 0 & 0 & -1 \\
		0 & 0 & 0 & 1 & 1 & 1 \\
		0 & 0 & 0 & 1 & -1 & 0 \\
		0 & 0 & 0 & 0 & 0 & 1
		\end{pmatrix}$
	
	
	Allgemein gilt für $(\mathbb{T}_n, I)$, so 
	$\mu(a, b) = \begin{cases}
	(-1)^r & \text{, falls } a \mid b \text{ und } \frac{b}{a} \text{ Produkt von } r \text{ verschiedenen Primzahlen.} \\
	0 & sonst 
	\end{cases}$
	
	In der Zahlentheorie : 
	$\mu(m) := \mu(1, m) = \mu(a, b)$ falls $a\mid b, \frac{b}{a} = m$
			
			
\end{enumerate}
	
	\subsection{Satz(Möbiusinversion)}
	$(M,\leq)$ geordnete Menge, $f: M\rightarrow \C$ beliebige Funktion.
	
	\begin{enumerate}
		\item 	$g:M \rightarrow \C$ sei definiert durch $g(x) = \sum_{\underset{y\leq x}{y \in M} } f(y)$
		
		Dann ist $f(x) = \sum_{\underset{y\leq x}{y \in M}}	g(y) * \mu (y,x)$ für alle $x \in M$
		
		\item 	$h: M \rightarrow \C$ sei definiert durch $h(x)= \sum_{\underset{y\leq x}{y \in M}}f(y)$ Dann: 	$f(x)= \sum_{\underset{y\leq x}{y \in M}} h(y)*\mu (x,y)$ für alle $x \in M$ 
		
	\end{enumerate}
	
	\subsubsection*{Beweis}
	\begin{enumerate}
		\item Sei $x \in M$.
		
		$\sum_{\underset{y\leq x}{y \in M}} g(y) \mu (y,x) = \sum_{\underset{y\leq x}{y \in M}}(\sum_{\underset{z\leq y}{z \in M}}f(z))\mu (y,x) = \sum_{\underset{y\leq x}{y \in M}}\sum_{\underset{z\leq y}{z \in M}} f(z) * \mu(y,x) $
		
		$\overset{(\star)}{=} \sum_{\underset{y\leq x}{y \in M}} \sum_{z \in M} f(z) * \zeta (z,y) * \mu (y,x) = \sum_{z \in M}f(z) \sum_{\underset{y\leq x}{y \in M}} \zeta (z,y)*\mu(y,x)$
		
		$= \sum_{z\in M} f(z)*\delta(z,x)=f(x)$
		
		$(\star): \zeta(z,y) = 1, \qquad z \leq y$
		
		$\qquad \zeta(z,y) = 0, \qquad z \nleq 0$
		\item analog
	\end{enumerate}
	
		\subsection{Beispiel} %3.10
		
		Zur Lösung einer Aufgabe seien die Lösungen von Teilaufgaben erforderlich, für deren Lösung wieder die Lösungen gewisser Teilaufgaben etc.
		
		Das führt zu partieller Ordnung auf der Menge aller (Teil-) Aufgaben $t_1, \dots, t_n$
		\\ $t_i \preceq t_j \Leftrightarrow$ Lösung von $t_i$ ist zur Lösung von $t_j$ erforderlich.
		
		Z.B. folgendes zugehörige Hasse-Diagramm.
		%TODO Diagramm??
		
		[...]
		
		Bekannt sei die Zeit $g(t_i)$ zur Lösung von $t_i$ (einschließlich der Zeiten für die dazu benötigten Teilaufgaben).
		
		Wie groß ist die ''reine'' Zeit $f(t_i)$ zur Lösung von $t_i$ ohne die Zeiten für die Teilaufgaben?
		
		$f(t_6) = ?$
		
		3.9 %TODO ref
		$f(t_6) = g(t_6)\mu(t_6, t_6)
		+ g(t_4)\mu(t_4, t_6) 
		+ g(t_3)\mu(t_3, t_6)
		+ g(t_2)\mu(t_2, t_6)
		+ g(t_1)\mu(t_1, t_6) $
		
		$\mu(t_2, t_6) = \mu(t_3, t_6) = \mu(t_4, t_6) = -1$ (3.7c) %TODO ref
		\\$\mu(t_1, t_6) = -\mu(t_1, t_1) - \mu(t_1, t_2) - \mu(t_1, t_3) - \mu(t_1, t_4) = -1 + 1 + 1 +1 = 2$
		
		$f(t_6) = 22 \cdot 1 - 11 \cdot 1 - 9 \cdot 1 - 10 \cdot 1 + 6 \cdot 2 = 4$	
		
		