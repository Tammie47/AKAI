\subsection*{Erinnerung} %TODO Einordnung. Im Skript von 2014 gehört der Teil zum nächsten Satz

$(R_h)$ $x_n = c_1x_{n-1} + \dots + c_k x_{n-k}$, $n > k$
\\ $ t^k -c_1t^{k-1} - \dots - t_k = 0 $ (*)

$d_1, \dots, d_s$ verschiedene Nullstellen von (*)

mit Vfh. $m_1, \dots, m_s$. $m_1 + \dots + m_s = k$

Allgemeine Lösung : 
\\$x_n = s_{1, 1}d_1^{n-1} + s_{1, 2}d_1^{n-1} + \dots + s_{1, m_1}n^{m_1-s}d^{n-1}
\\ + \dots \dots
\\ + s_{s,1}d_s^{n-1} + s_{s, 2}d_s^{n-1} + \dots + s_{s, m_s}n^{m_s-1}d_s^{n-1}
$

$s_{i, j}$ hängen von Anfangswerten ab.

\subsection{Satz}
$(R_k)$ wie oben, ebenso, $d_1,...,d_s, m_1,...,m_s.$

Sei o.B.d.A. $d:= \abs{d_1}=...=\abs{d_r}>\abs{d_{r+1}}\geq...\geq \abs{d_s}$

Sei $m = max\{m_1,...,m_r\}$

Ist $(a_1,a_2,...) $ eine Lösung von $(R_k)$, so ist 

$$ \abs{a_n} = \mathcal{O}(n^{m-1}*d^n)$$

(D.h. es ex. Konstante $C>0$ mit $\abs{a_n} \leq C*n^{m-1}d^n$ für alle $n \geq 1)$

\begin{tabular}{l l}
\underline{Speziell:}& Ist d<1, si ist $\underset{n\leftarrow \infty}{lim} a_n = 0$\\
& Ist d=1, so wächst $a_n$ höchstens polynomial\\
& Ist d>1, so wächst $a_n$ höchstens exponentiell
\end{tabular}

Genaues Wachstum hängt von Anfangswerten ab.

\underline{Ein Typ inhomogener lin.Rekursionen:}

$(R) x_n = c_1x^{n-1}+...+c_kx^{n-k}+a*r^n$ \qquad für alle n>k

$a,r$ Konstanen, $a\neq , \neq r$

(Wichtiger Spezialfall: $r=1$)

Lösungsmenge von $(R)$: Lösungsraum von zug. hom. Rek.$(R_k)$ + \underline{spez. Lösung von $(R)$}

Ansatz für spez. Lösung: $(dr,dr^2,dr^3,..., \underset{ \underset{n} { \uparrow }}{dr^n},...) \qquad d \neq 0$

\begin{tabular}{l l}
$(dr,dr^2,...)$ ist Lösung &$\Leftrightarrow dr^n = c_1+d+r^{n-1}+...+c_k+d+r^{n-k}+ar^n \qquad \forall n> k$\\
& $\Leftrightarrow \frac{(d-a)*r^n}{d}=c_1r^{n-1}+...+c_kr^{h-k} \qquad | :r^{n-k}$\\
& $\Leftrightarrow r^k - \frac{a}{d}r^k=\frac{(d-a)}{d}r^k=c_1+r^{k-1}+...+c_{k-1}r+c_k$ \\
& $\frac{a}{d} r^k = \underbrace{r^k-c_1r^{k-1}-...-c_{k-1}r-c_k}_{=:b}$
\end{tabular}
 
\subsection{Satz}
Falls $b\neq 0$ (d.h. r ist keine Nullstelle der char. Gleichung), so setze

$$ d:= \frac{a}{b}*r^k$$

mit diesem d hat man spezielle Lösung $(dr,dr^2,...)$ von $(R)$


\subsection{Beispiel : Türme von Hanoi (2.2.b)} %2.10 
%TODO ref

$a_n = 2a_{n-1} + 1$, $n \geq 2, a_1 = 1$

Rekursion ist vom Typ 2.9: %TODO ref
$a = 1, r = 1$

Charakteristische Gleichung : $t-2 = 0$
\\Allgemeine Lösung zur zugehörigen Rekursion $(R_h)$:
$\aset{(s, s\cdot 2, s\cdot2^2, s \cdot2^3, \dots) | s \in \C}$

Spezielle Lösung von $(R)$ nach 2.9. %TODO ref
($r=1$ ist keine Nullstelle von charakteristischer Gleichung)
\\ $b = -1, d = -1$
\\ Spezielle Lösung : $(-1, -1, -1, -1, \dots)$

Allgemeine Lösung der inhomogenen Rekursion: 
\\$(s, s\cdot 2, s\cdot2^2, s\cdot2^3, \dots) + (-1, -1, -1, \dots)
= (s-1, 2s-1, 2^2s-1, \dots, 2^{n-1}s-1, \dots)$

Bestimme $s$ mit $s-1 = a_1 = 1$, $s=2$

$(1, 3, 7, \dots)$ \qquad
$a_n = 2^{n-1}\cdot2-1=2^n-1$















