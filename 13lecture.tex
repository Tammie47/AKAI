\subsection{Definition} %3.11

Sei $(M, \preceq)$ geordnete Menge.

\begin{enumerate}
	\item $K \subseteq M$ heißt \emph{Kette}, wenn $(K, \preceq)$ total geordnet ist.
	
	\item $K \subseteq M$ heißt \emph{Antikette}, falls für alle $a, a' \in A, a \neq a'$ gilt:
	$ a \npreceq a'$ und $a' \npreceq a$ 
	
\end{enumerate}

\subsubsection*{Beispiel}

$(M, \preceq)$
%TODO zeichnen? 

[...]

$\aset{a, b, c, g, i}$ Kette, auch $\aset{a, d, i}$

$\aset{f, g, j}$ Antikette

\subsection{Bemerkung} %3.12

%TODO Thomas part 

\subsection{Satz} %3.13

$(M, \preceq)$ endliche geordnete Menge. Hat die größte Kette von $M$ Größe $k$, so kann $M$ mit $k$ Antiketten überdeckt werden (und nicht mit weniger (3.12b)) %TODO ref

\subsubsection*{Beweis}

Sei $x \in M$. Definiere Höhe $h(x)$ von $x$ als Maximalanzahl der Elemente $\neq x$ einer Kette, in der $x$ das größte Element ist. 

D.h. $h(x) = 0 \Leftrightarrow x$ hat keinen Vorgänger.

$A_i = \aset{x \in M : h(x) = i}$ Also $A_i = \emptyset$, falls $i \geq k$, da $k$ maximale Kettenlänge.

Daher $M = A_0 \cup \dots \cup A_{k-1}$, denn jedes Element hat Höhe.

Jedes $A_i$ ist Antikette : Angenommen nicht. Dann existieren $x, y \in A_i, x \neq y, x \prec y$

$h(x) = i$. Es existiert Kette $K$ mit $i+1$ Elementen, $x$ ist das Größte. $K \cup \aset{y}$ ist Kette mit $i+2$ Elementen, $y$ ist das größte Elemente, also $h(y) \geq i+1$, $\lightning$ zu $y \in A_i$

\subsection{Satz von Dilworth} %3.14
%TODO Thomas part






















