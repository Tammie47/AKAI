\documentclass[a4paper, 11pt, twosite, german, titlepage]{scrartcl}

%TODO doku
% German Support
\usepackage[german]{babel}
\usepackage[utf8]{inputenc}
\usepackage[T1]{fontenc}
% Math stuff
\usepackage{amsmath}
% style
\usepackage{fancyhdr}
\usepackage{newtxtext,newtxmath}
\usepackage[margin=2.5cm]{geometry}
% graphics
\usepackage{pgf, tikz}
\usetikzlibrary{decorations.pathreplacing, shapes}

%TODO Komaskript benutzen
% pagestyle: header...
\pagestyle{fancy}
\renewcommand{\sectionmark}[1]{%
\markboth {#1}{}}
\fancyhead[L]{Algebraische und kombinatorische Anwendungen in der Informatik SoSe16} 
\fancyhead[R]{\leftmark}	


%forall, exists, sum
\let\oldforall\forall
\renewcommand{\forall}{\hspace{3pt}\oldforall}
\let\oldexists\exists
\renewcommand{\exists}{\hspace{3pt}\oldexists}
\let\oldsum\sum
\renewcommand{\sum}{\oldsum\limits}
% Zahlbereiche
\DeclareMathOperator{\R}{\mathbb{R}}
\DeclareMathOperator{\Q}{\mathbb{Q}}
\DeclareMathOperator{\N}{\mathbb{N}}
\DeclareMathOperator{\Z}{\mathbb{Z}}
\DeclareMathOperator{\C}{\mathbb{C}}
% Zeichen
\newcommand{\gdw}{\Leftrightarrow} %iff (genau dann wenn)
\newcommand{\vnull}{\mathcal{O}} %Nullvektor
%
\newcommand{\ntoinf}{\stackrel{n \to \infty}{\longrightarrow}}
\newcommand{\inflimn}{\lim\limits_{n \to \infty}}
\newcommand{\inflimx}{\lim\limits_{x \to \infty}}
% Abk.
\newcommand{\without}[1]{\backslash \{#1\}} %Mengendifferenz
\newcommand{\abs}[1]{\left|#1\right|} % Betrag
\newcommand{\aset}[1]{\left\{#1\right\}} % Menge (grosse Klammern)
% 
\newcommand{\doubleunderline}[1]{\underline{\underline{#1}}}
\newcommand{\emphh}[1]{\textbf{#1}}








	
% Aufzaehlungsstandart		
\renewcommand{\labelenumi}{\alph{enumi})}

%TODO Titelseite
\title{Algebraische und kombinatorische Anwendungen in der Informatik}

% Dokumentinfo
\usepackage[pdftex, pdftitle={AKAI SoSe 16}, pdfauthor={Thomas Sachs, Felicia Saar}]{hyperref}
\hypersetup{pdftitle={AKAI}, pdfnewwindow, colorlinks, linkcolor=black}

\begin{document}

\maketitle
%TODO Disclaimer
\tableofcontents
\newpage
% muss hier zu Beginn stehen bleiben! Verhindert bei neuen Absaetzen das Einruecken
\setlength{\parindent}{0pt}
\setlength{\parskip}{1ex plus 0.5ex minus 0.2ex}


%Inhalt


\subsection*{Inhalt}
\begin{itemize}
	\item Abzählmethoden
	\item Rekursionen
	\item geordnete Mengen, Scheduling
	\item Anwendungen der linearen Algebra, z.B. Information Retrieval
\end{itemize}

\section{Abzählmethoden}

\subsection[Kardinalität paarweise disjunkter Mengen und des karthesischen Produkts]{} 
\label{subsec:disjcup(a)+kard(b)}
$S_1, \dots, S_n$ endliche Mengen
\begin{enumerate}
	\item  Sind $S_1, \dots, S_n$ paarweise disjunkt, so 
	\[\abs{S_1 \cup \dots \cup S_n} = \sum_{i=1}^{n}\abs{S_i}\]
	\item 
	\footnote{$S_1 \times \dots \times S_n = \aset{(s_1, \dots, s_n) : s_i \in S_i}$; Ist  $s_1 = s_2 = \dots = s_n = s  $ so $\underset{\leftarrow n \rightarrow}{s \times \dots \times s =: S^n}$}
	\[ \abs{S_1 \times \dots \times S_n} = \prod_{i=1}^n \abs{S_i} 
	 \]
\end{enumerate}

\subsection[Beispiel : Kardinalität der Vereinigung paarweise disjunkter Mengen]{Beispiele}

\begin{enumerate}
	\item
	Es gibt $2^n$ Wörter der Länge $n$ über $\aset{0,1}$
	\\ Allgemein Alphabet $S$ der Größe $q$, so gibt es $q^n$ Wörter der Länge $n$ über $S$
	
	\item[c)] Wie viele 3-stellige Zahlen (im Dezimalsystem; ggf. mit führenden Nullen) gibt es, die mindestens eine $1$ enthalten?
	Dafür gibt es mehrere\textbf{} Möglichkeiten:
	\begin{enumerate} %TODO Zahlen  + Mögl.
		\item $S =$ Menge aller 3.st. Zahlen mit mindestens einer 1.
		
		$S_1 = \aset{s \in S : \text{ an erster Stelle von } s \text{ steht } 1}$
		\\$S_2 = \aset{s \in S : \text{ die erste 1 von } s \text{ steht an 2. Stelle}}$
		\\$S_3 = \aset{s \in S : \text{ die erste 1 von } s \text{ steht an 3. Stelle}}$
		
		$S = S_1\: \dot{\cup}\; S_2 \;\dot{\cup}\; S_3$ 
		
		$S_1 = \aset{1} \times \aset{0, \dots, 9} \stackrel{\ref{subsec:disjcup(a)+kard(b)}b}{\Rightarrow} 
		\abs{S_1} = 1 \cdot 10 \cdot 10 = 100$
		\\
		$S_2 = \aset{0, 2, \dots, 9} \times \aset{1} \times \aset{0, \dots, 9} 
		\stackrel{\ref{subsec:disjcup(a)+kard(b)}b}{\Rightarrow} 
		\abs{S_2} = 9\cdot 1 \cdot 10 = 90$
		\\
		$S_3 = \aset{0, 2 \dots, 9} \times \aset{0, 2, \dots, 9} \times \aset{1}
		\stackrel{\ref{subsec:disjcup(a)+kard(b)}b}{\Rightarrow} 
		\abs{S_3} = 9 \cdot 9 \cdot 1 = 81$
		
		\ref{subsec:disjcup(a)+kard(b)}a: \quad 
		$\abs{S} = \abs{S_1} + \abs{S_2} + \abs{S_3} = 271$
		
		\item 
		$T_i = \aset{s \in S : s \text{ enthält mindestens } i \text{ mal die 1}}, i = 1, 2, 3, S = T_1 \;\dot{\cup}\; T_2 \;\dot{\cup}\; T_3$ 
		
		\begin{itemize}		
		
		\item $T_1 = T_1^1 \;\dot{\cup}\; T_1^2 \;\dot{\cup}\; T_1^3$,
	    \; $T_i^j = \aset{s \in T_1 : 1 \text{ steht an Stelle } j}$
	   
	    $\abs{T_1^1} = \abs{T_1^2} = \abs{T_1^3} = 9 \cdot 9 \cdot 1 = 81, 
	    \quad \abs{T_1} = 3 \cdot 81 = 243$ (\ref{subsec:disjcup(a)+kard(b)}) 
	    
	    \item  $T_2 = T_2^1 \cup T_2^2 \cup T_2^3$, 
	    \; $T_2^j = \aset{s \in T_2 : \text{ an der Stelle } j \text{ steht keine 1}}$
	    
	    $\abs{T_2^1} = \abs{T_2^2} = \abs{T_2^3} = 9$
	    \quad $ \abs{T_2} = 3 \cdot 9 = 27$
	    
	   \item  $\abs{T_3} = 1$
	    
	    \end{itemize}
	    
	    $\abs{S} = \abs{T_1 } + \abs{T_2} + \abs{T_3} = 271$
	    
	    \item
	    $G = \aset{0, \dots, 9}^3, \abs{G} = 1000$
	    
	    $S = G \without{\underbrace{x \in G : x \text{ enthält keine 1}}_{\aset{0, 2, \dots 9}^3}}$
	    
	    $\abs{\aset{0, 2, \dots, 9}^3} \stackrel{\ref{subsec:disjcup(a)+kard(b)}b}{=} 
	    9 \cdot 9 \cdot 9 = 729$
	    
	    $\abs{S} = 1000 - 729 = 271$
	 		
	\end{enumerate}
	
\end{enumerate}

\subsection*{Techniken im Beispiel}
\begin{itemize}
	\item Zerlegen einer abzählbaren Menge in disjunkte Teilmengen
	\item Abzählen des Komplements
\end{itemize}

Wie lässt sich \ref{subsec:disjcup(a)+kard(b)}a 
verallgemeinern, falls die $S_i$ nicht notwendig disjunkt?

$n=2$ : \begin{tikzpicture}
	\draw (0,0) circle (.5cm) node[above]{$S_1$};
	\draw (.7,0) circle (.5cm) node[above]{$S_2$};	
\end{tikzpicture}
$\abs{S_1 \cup S_2} = \abs{S_1} + \abs{S_2} - \abs{S_1 \cap S_2}$

$n=3$ : \begin{tikzpicture}
	\draw (0,0) circle (.5cm) node[above]{$S_1$};
	\draw (.7,0) circle (.5cm) node[above]{$S_2$};
	\draw (.35,-0.7) circle (.5cm) node[below]{$S_3$};	
\end{tikzpicture}
$\abs{S_1 \cup S_2 \cup S_3} = \abs{S_1} + \abs{S_2} + \abs{S_3}
                             - \abs{S_1 \cap S_2} - \abs{S_1 \cap S_3} - \abs{S_2 \cap S_3}
                             + \abs{S_1 \cap S_2 \cap S_3}$
                             
Allgemein gilt:

\subsection{Einschließungs-Ausschließungsprinzip}
\label{subsec:einausschliessprinzip}

\[\abs{ S_1 \cup \dots \cup S_n} = \sum_{k=1}^n(-1)^{k+1} \sum_{1 \leq i_1 < i_2 < \dots < i_k \leq n}{\abs{S_{i_i} \cap \dots \cap S_{i_k}}} \]
(Beweis WHK, 2.32)

\subsection{Beispiel : Einschließungs-Ausschließungsprinzip}

\begin{enumerate}
	\item Wie viele 8-stellige Zahlen (Dezimalsystem, mit führenden Nullen) enthalten nicht alle ungerade Ziffern?
	
	$S =$ Menge aller 8-st. Zahlen.
	\\$S_i = \aset{s \in S : s \text{ enthält nicht } i}, i = 1, 3, 5, 7, 9 $
	
	$\abs{S_1 \cup S_3 \cup S_5 \cup S_7 \cup S_9}$
	
	\ref{subsec:einausschliessprinzip} anwenden 
	($n=5 $)
	
	\begin{tabular}{l@{ : }lllc}

	$k=1$ 
	& $\underset{=9^8}{\abs{S_1}}, \dots, \underset{=9^8}{\abs{S_9}}$
	&
	& $5 \cdot 9^8$ 
	& (+)\\
	
	
	$k=2$ 
	& $\underbrace{\abs{S_1 \cap S_3}}_{\text{8-st. ohne 1, ohne 3}} = 8^8$, entsprechend $\abs{S_1 \cap S_5}, \dots \abs{S_7 \cap S_9}$ 
	& 10 mal 
	& $10 \cdot 8^8$ 
	&(-) \\
	
	$k=3$ 
	& $\abs{S_1 \cap S_3 \cap S_5} = 7^8$ 
	& 10 mal 
	&$10 \cdot 7^8$ 
	&(+) \\
	
 	$k=4$
 	& $\abs{S_1 \cap S_3 \cap S_5 \cap S_7} = 6^8$ 
 	& 5 mal 
 	& $5 \cdot 6^8$ 
 	& (-) \\
 	
 	$k=5$ 
 	& $\abs{S_1 \cap S_3 \cap S_5 \cap S_7 \cap S_9} = 5^8$ 
 	& 1 mal 
 	& $5^8$ 
 	& (+) \\
	 \end{tabular}
 		
	
	$\abs{S_1 \cup S_3 \cup S_5 \cup S_7 \cup S_9} = 5 \cdot 9^8 - 10 \cdot 8^8 + 10\cdot 7^8 - 5 \cdot 6^8 + 5^8 = 97.102.000$
	
		
	(ungefähr) $97\%$ aller 8-stell. Zahlen enthalten nicht alle ungeraden Ziffern.
		
	\item Wie lautet die Antwort in a), wenn man s-stelligen Zahlen ($s\geq 8$) betrachtet?\\	
	Wie in a): Anzahl $= 5*9^s-10*8^s+10*7^s-5*6^s+5^s$\\
		
	Anzahl aller s-stelligen Zahlen: $10^s$\\
		
	Verhältnis: $5*(\frac{9}{10})^s-10*(\frac{8}{10})^s+10*(\frac{7}{10})^s-5*(\frac{6}{10})^s+(\frac{5}{5})^s$
		
	Anal.: $q^s \rightarrow 0,$ falls $s \rightarrow \inf$, $\abs{q}<1$
	$lim V_s = 0$
	$s\rightarrow \inf$  
\end{enumerate}
                             








































\end{document}