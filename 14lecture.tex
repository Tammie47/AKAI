\subsection{Definition} % 3.15

Sei $G = (E,K)$, $E$ Eckenmenge, $K$ Kantenmenge

($G$ einfacher Graph, d.h. keine Mehrfachkanten, keine Schleifen)

Jede Kante kann mit 2-elementiger Teilmenge von $E$ identifiziert werden.

\begin{enumerate}
	\item $G$ heißt \emph{bipartit}, falls es eine Zerlegung $E=S\: \dot{\cup}\; T$, $S, T$ disjunkt gibt, so dass jede Kante einen Endpunkt in $S$ und einen in $T$ hat.
	
	\item
	$V \subseteq E$ heißt \emph{vertex cover} (überdeckende Eckenmenge), falls jede Kante mindestens einen Endpunkt in $V$ hat.
	
	\item
	Ein \emph{Matching} ist Teilmenge $L \subseteq K$, so dass keine 2 Kanten in $L$ einen Endpunkt gemeinsam haben.
	
\end{enumerate}

\subsubsection*{Beispiel}

[...]

\underline{Klar:} Ist $m$ die Größe eines Matchings, $c$ die Größe eines vertex covers, so $m \leq c$
\\(Von jeder Kante des matchings mindestens ein Endknoten )

\subsection{Satz von König (1916)} %3.16
 %TODO Thomas


\subsection{Bemerkung}

3.13 \ 3.14 \ 3.16 Beispiele von Max-Min-Sätzen %TODO ref

Maximierung einer Größe ist äquivalent zu Minimierung einer anderen Größe.

Weiteres Beispiel : Max-Flow Min-Cut-Theorem.

Netzwerk : Gerichteter Graph (Kanten hoher Orientierung)
und jede Kante $k$ hat Gewicht $c(k) \geq 0, c(k \in \R$
($c(k) = $ Kapazität von $k$). 
Es gibt zwei ausgezeichnete Ecken $A$ und $Z$ ($A$ Anfangsknoten, $Z$ Zielknoten), bei $A$ gibt es nur ausgehende Kanten, bei $Z$ gibt es nur eingehende Kanten.

[...]

Ein \emph{Fluss} in einem Netzwerk ist Funktion $f$, die jeder Kante $k$ ein $f(k) \in \R$ zuordnet mit :

\begin{enumerate}
	\item
	$0 \leq f(k) \leq c(k)$ für alle $k$.
	
	
	\item 
	Für jeden Knoten $X$, $X \neq A, Z$, gilt:
	
	$\sum_{*_1}f(k) = \sum_{*_2}f(k')$
	\\ ($(*_1): k$ eingehende Kante bei $X$, $(*_2): k'$ ausgehende Kante bei $X$)
	
\end{enumerate}

Wert eines Flusses $f$ :
$\abs{f} = \sum_{*3}f(k)$
\\$((*_3): k$ ausgehende Kanten von $A$)

Ein \emph{Schnitt} $S$ ist eine Menge von Kanten, so dass jeder Weg von $A$ nach $Z$ mindestens eine Kante aus $S$ beinhaltet.
Kapazität von $S := \sum_{k \in S} c(k)$

\subsection*{Max-Flow Min-Cut-Theorem (Ford, Fulkerson)}

Max. Wert eines Flusses = Minimale Kapazität eines Schnittes.

(Beweis z.B. van Lint, Wilson, A Course in Combinatorics, Th. 7.1)
















