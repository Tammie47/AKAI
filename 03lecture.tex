\subsection{Satz}
\begin{enumerate}
\item $k,n \in \N_0$

Anzahl der Möglichkeiten aus einer Menge mit n Elementen \textit{k} Elemente auszuwählen ist:


\begin{tabular}{c|c|c}
	$k$ aus $n$ & ohne Berücksichtigung der Anordnung & mit Berücksichtigung der Anordnung\\ \hline
	ohne Wdhl. & $\binom{n}{k}$ & $[n]_k$\\ \hline
	mit Wdhl. & $\binom{n+k-1}{k} $& $n^k$
\end{tabular}


\item $k,n \in \N$

Anzahl der geordneten n-Tupel$(x_1,...,x_n)$

$x_i\in \N_0,$ mit $\sum_{i=1}^{n} = k$ ist $\binom{n+k-1}{k}$

\item $k,n \in \N$

Wie b), aber $x_i \in \N$; Anzahl ist $\binom{k-1}{n-1}$

$ \Biggl( \sum_{i=1}^{n}(x_i-1)=k-n$ \quad Nach b): Anzahl $\binom{n+(k-n)-1}{k-n} = \binom{k-1}{k-n}=\binom{k-1}{k-1-(k-n)} = \binom{k-1}{n-1} \Biggr) $ 
\item Nach a): Anzahl der k-elementigen Teilmengen einer Menge mit \textit{n} Elementen ist $\binom{n}{k.}$

Anzahl aller Teilmengen einer Menge mit n Elementen:
$$ \sum_{k=0}^{n}\binom{n}{k}= (1+1)^n = 2^n$$ (Mathe I)
\end{enumerate}

\subsection{Beispiele}

\begin{enumerate}
	\item
	52 Karten, je 13 von Karo, Herz, Pik, Kreuz.
	Wie groß ist die Wahrscheinlichkeit 5 Karten von einer der vier Sorten zu bekommen?
	
	$\frac{\text{Anzahl der günstigen Fälle}}{\text{Anzahl aller Fälle}}
	= \frac{4 \cdot \binom{13}{5}}{\binom{52}{5}}
	\approx 0.00198$
	
	\item 
	Wahrscheinlichkeit, genau 3 Asse zu bekommen (bei 5 Karten):
	
	$\frac{\binom{4}{3} \cdot \binom{48}{2}}{\binom{52}{5}} \approx 0.001736$
	
	\item
	Wahrscheinlichkeit mind. 3 Asse zu erhalten:
	
	$\frac{\binom{4}{3} \cdot \binom{48}{2} + 48}{\binom{52}{5}} \approx 0.001755$
	
	\item Wie viele Möglichkeiten gibt es 12 identische Kugeln in 8 unterschiedliche Kisten zu verteilen (Kisten dürfen leer bleiben)? 
	
	$x_i = $ Anzahl der Kugeln in Kiste $i$. $x_i \in \N_0$
	
	$\sum_{i=1}^{8}x_i = 12$. 1.9b %TODO ref
	anwenden mit $n=8, k=12 $ :
	$\binom{19}{12} = \binom{19}{7} = 50388$
	
	\item
	Wie d) %TODO ref?
	aber keine Kiste darf leer bleiben: 1.9.c; %TODO ref
	$\binom{11}{7} = 330$
	
	
\end{enumerate}


\subsection{Satz}
Anzahl der Möglichkeiten k identische/unterschiedliche Objekte auf \textit{n} identische/unterschiedl. Kisten zu verteilen (Kisten dürfen leer bleiben):

\begin{tabular}{c|c|c}
	& \textit{n} ident. Kisten & \textit{n} untersch. Kisten \\ \hline
	\textit{k} ident. Obj. &$p_n(k)$ &$\binom{n+k-1}{k})$ \\ \hline
	\textit{k} untersch. Obj. & $\sum_{i=1}^{n} S(k,i)$&$n^k$ 
\end{tabular}

\subsection{Definition}
\begin{enumerate}
	\item \textit{A} Menge. \underline{Partition} von \textit{A}: $\{ A_1,...,A_r \},~ A_i \subseteq A~ A_i\cap A_j = \emptyset$ für $i\neq j$ und $A = A_1 \cup ... \cup A_r$ 
	\subsubsection*{Beispiel:}
	$\abs{A}= 4$, $r=2$
	
	$\{\{1\}, \{2,3,4\}\},~ \{\{2\}, \{1,3,4\}\}~ \{\{3\}, \{1,2,4\}~ \{\{4\}, \{1,2,3\}\}\}$
	$\{\{1,2\}, \{3,4\}\},~\{\{1,3\}, \{2,4\}\},~\{\{1,4\}, \{2,3\}\}$ 7 Partitionen in zwei Teilmengen.
	
	\item Anzahl aller Partitionen einer Menge mit \textit{k} Elementen in r Teilmengen = 
	
	$$S(k,r)$$
	\begin{center}
	(Stirlingzahl 2.Art) \qquad \qquad  (James Stirling, 1692-1770)
	\end{center}
	
	\item Anzahl aller Partitionen einer Menge mit \textit{k} Elementen:
	
	$$ B_k = \sum_{i=1}^{k} S(k,i) \qquad \underline{Bell-Zahlen}$$
	\begin{center}
		(E.T.Bell,1883-1960)
	\end{center}
	
	\item $P_t(k) = $ Anzahl der Zerlegungen von $k\in \N_0$ in r Summanden aus $\N_0$ ohne Berücksichtgung der Reihenfolge der Summanden.
	
	$=\abs{\{(x_1,...,x_r): x_i \in \N_0, x_1 \geq x_2 \geq ... \geq x_r, \sum_{i=1}^{r} x_i=k\}} $
	
	\begin{tabular}{c c c c c}
	\underline{\textbf{Partitionszahlen Beispiel:}}	&	$P_2(4)=3$:& 4+0 & $P_3(4)=4$	& 4+0+0 \\
												 	&  			& 3+1    &				& 3+1+0 \\
													&   		& 2+2 	 & 				& 2+2+0 \\ 
													& 			&		 &				& 2+1+1 \\
	\end{tabular}


\end{enumerate}

\subsection*{Beweis von 1.11} %TODO ref

\begin{enumerate}
	\item
	$k$ unterschiedliche Objekte, $n$ unterschiedliche Kisten. 
	
	Für jedes Objekt $n$ Möglichkeiten. Insgesamt $n^k$.
	
	\item
	$k$ identische Objekte, $n$ unterschiedliche Kisten:
	Wie in Beispiel 1.10.d), %TODO ref
	(Anwendung von 1.9.a): %TODO ref
	$\binom{n+k-1}{k}$	
	
	\item
	$k$ unterschiedliche Objekte, $n$ identische Kisten 
	
	Anzahl = Anzahl aller $\underset{\text{Kisteninhalte}}{\text{Partitionen}}$ %TODO  aufräumen ;)
	 einer Menge von $k$ Elementen in maximal $n$ Teilmengen
	 $= \sum_{i=1}^nS(k, i)$
	 
	 \item
	 $k$ identische Objekte, $n$ identische Kisten. 
	 
	 Verteilung ist bestimmt durch: eine Kiste enthält $x_1$ Objekte, eine Kiste enthält $x_2$ Objekte, \dots, eine $x_n$ Objekte.
	 $\sum_{i=1}^{n}x_i=k, x_i \in \N_0$
	 
	 Anzahl: $P_n(k)$
	
\end{enumerate}












