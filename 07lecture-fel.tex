% TODO Thomas stuff

\subsection{Satz} % 2.4


Gegeben $(R_h) : x_n = c_1x_{n-1} + \dots + c_kx_{n-k}$ für $n > k$

\begin{enumerate}
	\item Ist $d\in\C$ eine Lösung der charakteristischen Gleichung 
	\[ (*) t^k - c_1t^{k-1} - \dots - c_{k-1}t - c_k = 0 \]
	so ist $w = (1, d, d^2, \dots)$ Lösung von $(R_h)$
	
	\item
	Besitzt (*) $k$ verschiedene Lösungen $d_1, \dots, d_k$, so bilden die $w_i = (1, d_i, d_i^2, \dots)$, $i = 1, \dots k$, Basis des Lösungsraums von $(R_h)$
	
	\item
	Unter den Vor. in b) sei $a = (a_1, \dots, a_k) \in \C^k$ ein Vektor von Anfangswerten.
	
	Seien $s_1, \dots, s_k$
 die Lösungen des LGS
 
	$\begin{array}{lcl}
	s_1 + \dots + s_k &=& a_1 \\
	d_1s_1 + \dots + d_ks_k &=& a_2 \\
	\vdots \\
	d_1^{k-1}s_1 + \dots + d_k^{k-1}s_k &=& a_k
	\end{array}$
	
	so ist $l(a) = (s_1 + \dots + s_k, s_1d_1 + \dots + s_kd_k, \dots, s_1d^{n-1} + \dots + s_kd_k^{n-1}, \dots)$

	d.h. \[ a_n = s_1d_1^{n-1} + \dots + s_kd_k^{n-1} \text{ für alle } n \in \N \]
	(\emph{geschlossene Form})
	
	
\end{enumerate}

\subsection{Beispiel} %2.5

\begin{enumerate}
	\item Fibonacci-Folge: 
	$F_n = F_{n-1} + F_{n-2}, n \geq 3, F_1 = 1, F_2 = 2$ (2.2.a) %TODO ref
	
	Charakteristische Gleichung 
	$t^2 - t - 1 = 0$
	\\ Lösungen : $d_{1,2} = \frac{1 \pm \sqrt{5}}{2}$
	
	$s_1 + s_2 = 1$
\\	$s_1\left(\frac{1 \pm \sqrt{5}}{2}\right) 
	+ s_2\left(\frac{1 - \sqrt{5}}{2}\right) = 2$
	
	$s_1 = \frac{1}{\sqrt{5}}\left(\frac{3 + \sqrt{5}}{2}\right)
	= \frac{1}{\sqrt{5}} \left( \frac{1 + \sqrt{5}}{2} \right)^2$
	\\ 
	$s_2= \frac{1}{\sqrt{5}} \left( \frac{-3 + \sqrt{5}}{2} \right) 
	= \frac{1}{\sqrt{5}} \left( \frac{1 - \sqrt{5}}{2} \right)^2$
	
	$F_n = s_1d_1^{n-1} + s_2d_2^{n-1}$
	%
	\[ F_n = \frac{1}{\sqrt{5}}\left( \frac{1 + \sqrt{5}}{2} \right)^{n+1} + \frac{1}{\sqrt{5}}\left( \frac{1 - \sqrt{5}}{2} \right)^{n+1} \]
	%
	(Manchmal wird auch definiert:
	$ F_n = \frac{1}{\sqrt{5}}\left( \frac{1 + \sqrt{5}}{2} \right)^n 
	+ \frac{1}{\sqrt{5}}\left(\frac{1 - \sqrt{5}}{2}\right)^n
	\\ F_1 = 1, F_2 = 1 $)
	 
	$\abs{\frac{1 - \sqrt{5}}{2}} < 1$, mit wachsendem $n$ geht $\frac{1}{\sqrt{5}}\left( \frac{1 - \sqrt{5}}{2} \right)^{n+1}$ gegen 0.
	
	Tatsächlich $F_n$ ist die nächste ganze Zahl zu $\frac{1}{\sqrt{5}}\left( \frac{1 + \sqrt{5}}{2} \right)^{n+1} \approx \frac{1}{\sqrt{5}}1.618^{n+1} $
	
	$\frac{1 + \sqrt{5}}{2}$ Zahl des goldenen Schnitts.
	
	$|---\underset{M}{-}---|--\underset{m}{-}--| \qquad \frac{M}{m} = \frac{M + m}{M} = \frac{1 + \sqrt{5}}{2}$
	
	
	\item 
	$x_n = 2x_{n-1} - x_{n-2} + 2x_{n-3}, n \geq 4, x_1 = 0, x_2 = 4, x_3 = 10 $

	$t^3 - 2t^2  + t - 2 = 0$
	\\ $d_1 = 2$
	\\ $t^3 - 2t^2 + t - 2 = (t-2)(t^2+1)$
	\\ $d_2 = i$, $d_3 = -i$
	

	$s_1 + s_2 + s_3 = 0$
	\\$2s_1 + is_2 - is_3 = 4$
	\\$4s_1 -s_2 -s_3 = 10$	
	
	$s_1 =2, s_2 = -1, s_3 = -1$
	
	$ x_n = 2 \cdot 2^{n-1} - i^{n-1} - (-i)^{n-1}
	      = \begin{cases}
	      2^n & n \text{ gerade} \\
	      2^n - 2 & n \equiv 1  \;(\!\!\!\!\mod 4) \\
	      2^n + 2 & n \equiv 3 \;(\!\!\!\!\mod 4)
	      \end{cases} $
	
	 

\end{enumerate}



\subsection{Satz} %2.6
%TODO Thomas stuff

\subsection{Beispiel} % 2.7

$x_n = 3x_{n-2} -2x_{n-3}, n \geq 4, x_1 = 1, x_2 = 2, x_3 = 6$

Charakteristische Gleichung $t^3-3t+2 = 0$, 
\\ $d_1 = 1$ (2 mal), $m_1 =2$
\\$t^3 -3t + 2 = (t-1)^2(t+2)$
\\$d_2 =2$ (1 mal), $m_2 = 1$

$w_{1,1} = (1, 1, 1, 1, \dots)$
\\ $w_{1, 2} = (1, 2, 3, 4, \dots)$
\\ $w_{2, 1} = (1, -2, 4, -8, \dots, (-2)^{n-1}, \dots)$

$(x_1, x_2, x_3, \dots) = s_1w_{1, 1} + s_2w_{1, 2} + s_3w_{2,1}$

$s_1 + s_2 + s_3 = 1$
\\$s_1 + 2s_2 -2s_3 = 2$ 
\\$s_1 + 3s_2 + 4s_3 = 6$

Lösung: $s_1 = - \frac{4}{3}, s_2 = 2, s_3 = \frac{1}{3}$

$x_n = -\frac{4}{3} + 2n + \frac{1}{3}(-2)^{n-1}$


























